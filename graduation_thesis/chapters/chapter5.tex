\chapter{编队控制飞行实验}%TODO:仿真需要给出控制器的相关参数,以及初始条件等等。
\label{chap:simulatin_expermient}
上一章介绍了无人机编队控制器的仿真环境,并基于MATLAB/Simulink
工具以及Gazebo仿真环境分别分别完成了数学仿真以及动力学仿真,验证了
前文所设计编队控制器的合理性。
本章中,则从双机编队的实物飞行实验的角度出发,介绍双机编队的软硬件
环境选配、硬件连接、通讯搭建等方面介绍固定翼无人机双机编队的硬件解决方案
以及飞行试验过程。
\section{无人机软硬件环境选配}
本文所设计的编队控制器是以开源自动驾驶仪PX4的内环为基础的,PX4的内环自动驾驶仪运行在Pixhawk这一开源硬件之上,成为上层控制的下位机(slave computer):
编队控制算法运行在具有ROS环境的上位机(host computer)中;
整体的软件硬件选配关系如下图所示:
\begin{figure}[H]
    \centering
    \includegraphics[width=1\textwidth]{figures/c4/c4-soft-hard.png}
    \caption{硬件软件选配关系}\label{fig:c4-soft-hard.png}
\end{figure}
上位机(host computer)的选择主要由其性能决定,应满足ROS基本环境的正常运行以及编队控制算法的需求;其次应考虑该硬件的寿命,体积,工况
要求等指标。下位机(slave computer)是PX4等算法运行的介质,也是飞行之中的重要传感器如惯性原件(IMU)、磁罗盘以及定位模块(GPS Module)的
工作平台,选择时考虑其传感器精度,平台计算能力等因素;无人机是编队控制的载具平台,应根据上述硬件以及必要航电设备选择翼面积、起飞质量
有效载荷等重要参数;根据硬件安放位置选择合适的机舱外形;根据编队控制需要确定平飞速度;根据期望推力选择发动机型号;根据起飞降落方式
选择起落架类型。必要时,应根据性能要求以及指标设计无人机。
\section{双机编队通讯连接}
在实现无人机编队过程中,双机通信必不可少。通信模块相对于无人机系统相对独立,在此只介绍硬件实现的一种手段:通讯模块选用P900芯片,通讯协议
选用mavlink协议,驱动由ROS下的串口功能包“serial”提供;领机将自身的状态信息通过串口从其自动驾驶仪滤波之后的值中获取
,之后进行mavlink的编码,最后通过数传模块发送之后,接收端进行解包;实际双机编队飞行试验的硬件链接逻辑如下图所示:
\begin{figure}[H]
    \centering
    \includegraphics[width=1\textwidth]{figures/c4/double_plane_real}
    \caption{双机编队硬件链接}\label{fig:c4-double_plane_real}
\end{figure}
\section{双机编队实验安全保障机制}
\section{双机编队实验设计}
\section{本章小结}
