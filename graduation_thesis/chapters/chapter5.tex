\chapter{无人机编队整体控制逻辑、仿真环境以及硬件选型}
\label{chap:hardware}
上一章中完整地介绍了编队控制器的设计方法,以及实际应用时的改进;
本章主要介绍无人机编队的编队控制算法之外的系统组成部分;之后介绍无人机编队的整体的控制的实现逻辑,之后将介绍无人机编队的动力学仿真环境的搭建。
最后将介绍本次设计之中所用到的无人机型号,自动驾驶仪硬件以及姿态自动驾驶仪内环基本控制逻辑。
\section{ 软件控制逻辑以及软件环境 }
编队控制算法所运行的软件环境是ROS(Robot Operating System, 机器人操作系统)。ROS是一个为机器人开发而设计的
开源类操作系统平台。本质上,ROS是一个运行于实际操作系统上的应用程序,因而与实际的操作系统有着本质的区别,
但是其对于硬件的抽象以及应用层程序之间的通信的支持使其具有
操作系统的一些重要特征。ROS提供的进程间的通讯方式多种多样,针对于不同要求的应用场景
则设计了不同的进程间的通讯方式,例如消息(Message)、服务(Service)和参数(Param)等;
由于ROS开源社区的蓬勃发展,越来越多的第三方库可以供给开发者使用,也使得针对无人机的
应用开发变得十分方便。本次使用的应用程序接口是
ROS下的mavros功能包,
本功能包的作用是:将来自自动驾驶仪的无人机状态数据由mavlink通信协议转换为ROS的进程间的通讯的协议;
将来自编队控制器的姿态驾驶仪内环的期望姿态角以及期望油门值按照mavlink的协议进行编码,从而起到沟通编队控制器以及姿态驾驶仪内环的桥梁作用。
这里涉及到的Mavlink通讯协议,最早由苏黎世联邦理工大学的研究人员开发,是一种十分轻量级的消息传递协议,广泛用于当今的无人机通讯
,以及无人机内部组件以及无人机与地面站之间的通讯。
MAVLink遵循现代的混合发布-订阅和点对点设计模式:数据流作为主题发送/发布,而配置子协议(如任务协议或参数协议)是点对点重传。
MAVlink通讯已经作为一个专用库封装和发布,通讯时只需要调用相应的库函数进行打包以及包解析,无需再自行定义消息传输格式。
自动驾驶仪则使用第\ref{chap:formation_dynamic_equ}章中介绍的PX4开源自驾仪,此处不再赘述。
\section{无人机软硬件环境选配}
本文所设计的编队控制器是以开源自动驾驶仪PX4的内环为基础的,PX4的内环自动驾驶仪运行在Pixhawk这一开源硬件之上,成为上层控制的下位机(slave computer):
编队控制算法运行在具有ROS环境的上位机(host computer)中;
整体的软件硬件选配关系如下图所示:
\begin{figure}[H]
    \centering
    \includegraphics[width=1\textwidth]{figures/c4/c4-soft-hard.png}
    \caption{硬件软件选配关系}\label{fig:c4-soft-hard.png}
\end{figure}
上位机(host computer)的选择主要由其性能决定,应满足ROS基本环境的正常运行以及编队控制算法的需求;其次应考虑该硬件的寿命,体积,工况
要求等指标。下位机(slave computer)是PX4等算法运行的介质,也是飞行之中的重要传感器如惯性原件(IMU)、磁罗盘以及定位模块(GPS Module)的
工作平台,选择时考虑其传感器精度,平台计算能力等因素;无人机是编队控制的载具平台,应根据上述硬件以及必要航电设备选择翼面积、起飞质量
有效载荷等重要参数;根据硬件安放位置选择合适的机舱外形;根据编队控制需要确定平飞速度;根据期望推力选择发动机型号;根据起飞降落方式
选择起落架类型。必要时,应根据性能要求以及指标设计无人机。

在实现无人机编队过程中,双机通信必不可少。通信模块相对于无人机系统相对独立,在此只介绍硬件实现的一种手段:通讯模块选用P900芯片,通讯协议
选用mavlink协议,驱动由ROS下的串口功能包“serial”提供;领机将自身的状态信息通过串口从其自动驾驶仪滤波之后的值中获取
,之后进行mavlink的编码,最后通过数传模块发送之后,接收端进行解包;实际双机编队飞行试验的硬件链接逻辑如下图所示:
\begin{figure}[H]
    \centering
    \includegraphics[width=1\textwidth]{figures/c4/double_plane_real}
    \caption{双机编队硬件链接}\label{fig:c4-double_plane_real}
\end{figure}
\section{无人机编队动力学仿真环境}
所谓无人机动力学仿真环境,是在考虑无人机的动力学过程的基础之上搭建的仿真环境,相较于控制器的数学仿真,此种仿真环境考
虑了无人机作为一个实际的被控系统而存在的过渡过程,不确定性以及扰动因素,将更加还原无人机飞行时的实际状态。
本次动力学仿真环境基于Gazebo这一通用的开源仿真环境,除调用物理引擎仿真飞行器6自由度的动力学模型外,还可以产生相应的、添加噪声
污染的传感器数据反馈给下位机自动驾驶仪。
\begin{figure}[H]
    \centering
    \includegraphics[width=0.75\textwidth]{figures/c4/Gazebo.png}
    \caption{Gazebo仿真环境}\label{fig:c4-Gazebo}
\end{figure}

Gazebo仿真环境可看做由世界(World)、模型(Model)、传感器(Sensor)、物理引擎(Physical Engine)以及插件(Plugin)等模块
组成:世界(World)包含了仿真所使用的全部组件,例如模型(Model)、传感器(Sensor)、灯光(Light)等;模型(Model)的是构成
构成世界的组成部分,可以多次复用。传感器(Sensor)是一类特殊的模型,可以产生带有噪声的传感器数据信息。物理引擎(Physical Engine)
是驱动模型运动的组件,对应的物理库为基本仿真组件提供了一个简单和通用的界面,包括刚体、碰撞形状和表示关节约束的关节。
这个接口集成了四个开源物理引擎。

Gazebo在仿真时,首先加载世界,包括其中的各种参数如重力场定义、灯光等,以及定义了飞机4通道操纵机构飞机模型;
之后,各类插件,例如空气动力学插件、传感器插件以及环境插件等调用物理引擎接口,产生诸如无人机刚体运动学,传感器噪声
、大气环境等相关等相关仿真过程量。至此,通过Gazebo的GUI可以在可视化界面上得到无人机的运动图像。通过PX4官方给出的仿真工具,
可将在Gazebo中运行的固定翼无人机传感器的测量原始信息编码成MAVlink消息格式,反之亦然,之后,再通过TCP4560端口与PX4自驾仪进行通讯,即
Gazebo中的相应控制舵面运动的插件接受来自PX4的控制器信息,再将仿真的传感器数据进行打包和发送,从而构成了仿真环境。

总而言之,Gazebo考虑了无人机详细的动力学方程,只是在空气动力和力矩的产生方面,使用了简单的工程化的计算方法,使用速度、翼面积、
升阻系数等来计算无人机所受的力和力矩,无法模拟细致入微的空气动力学特性;因而不能使用Gazebo环境进行诸如油耗测试等涉及空气动力学
影响的仿真实验。但是,作为编队控制器设计与仿真阶段,Gazebo不失为一种强大的验证算法的手段。

另外,仿真之中的飞机的动力学模型由Gazebo仿真环境给出,可自定义飞机的质量,推力等参数;仿真之中的传感器数据由Gazebo产生,由PX4读取,作为
真实环境之中的传感器数据的仿真。上述参量均可以按照本次无人机平台做出相应的修改。

基于ROS的编队控制程序同时运行,通过mavros等程序API进行数据交互,完成动力学仿真。
相应的仿真程序之间的逻辑关系如下图所示:
\begin{figure}[H]
    \centering
    \includegraphics[width=0.75\textwidth]{figures/c4/px4_sitl_overview.png}
    \caption{编队控制仿真逻辑}\label{fig:px4_sitl_overview}
\end{figure}
在启动双机仿真时,启动脚本会加载两个无人机模型,并将与其相关的ROS消息用形如“UAV0、UAV1”等编号形式进行区分。
%TODO:需要添加Gazebo的动力学模型是如何载入进来的。
\section{本章小结}
本章着重介绍了本次设计所需要用到的仿真软件环境,详细介绍了Gazebo的软件架构,仿真机理以及仿真的相关特性。并通过ROS操作系统以及
MAVROS功能包搭建了与PX4内环进行数据交互的软件接口;之后结合Gazebo建立起了完整的编队控制器的无人机动力学仿真实验环境,
为后文的动力学仿真提供了相应的平台。
